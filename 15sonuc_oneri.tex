\section{SONUÇLAR VE ÖNERİLER}

Bu çalışma, konteyner teknolojilerinin yaygınlaşmasıyla birlikte ortaya çıkan karmaşıklığı azaltmak amacıyla geliştirilmiştir. Konteyner kullanımını kolaylaştırmayı hedefleyen bu çalışmade, konteyner oluşturma, çalıştırma ve sonuçları başarıyla gerçekleştirilmektedir. 

çalışmanin başarılı sonuçları aşağıdaki şekillerde değerlendirilebilir:
\begin{itemize}
    \item Konteyner İşlemleri: çalışma, Docker Engine API ile entegre olarak konteyner oluşturma, çalıştırma, durdurma ve silme gibi işlemleri başarılı bir şekilde gerçekleştirmektedir. Bu sayede kullanıcılar, Docker konteynerlerini kolaylıkla yönetebilmektedir.
    \item Çalışma Performansı: çalışmanin uygulama performansı, Guzzle kütüphanesinin etkili kullanımı ve Docker Engine API'nin sağladığı hızlı yanıtlar sayesinde yüksek seviyede tutulmaktadır. Bu da kullanıcı deneyimini olumlu yönde etkilemektedir.
    \item Theharvester Servisi: çalışma, servislerin eklenmesini desteklemektedir. Örnek olarak Theharvester servisi geliştirilmiş ve başarıyla çalışmaye entegre edilmiştir. Bu sayede kullanıcılar, çalışmalerine farklı servisleri ekleyerek genişletme imkanına sahip olmaktadır.
\end{itemize}




\subsection*{Öneriler}
\begin{itemize}
    \item Kullanıcı Arayüzü Geliştirme: Gelecekte, web tabanlı bir kullanıcı arayüzü oluşturulabilir. Bu arayüz sayesinde kullanıcılar, konteyner oluşturma, silme, güncelleme ve sonuçları inceleme gibi işlemleri görsel bir şekilde gerçekleştirebilir. Bu, kullanım kolaylığı sağlayacak ve çalışmanin kullanıcı tabanını genişletecektir.
    \item Hata Yönetimi ve Güncelleme: çalışma, hataların ve istisnaların yönetimini etkin bir şekilde gerçekleştirmektedir. Gelecekte, kullanıcı geri bildirimleri ve testlerle çalışmanin daha da güçlendirilmesi ve hata sayısının en aza indirgenmesi sağlanabilir. Ayrıca, Docker Engine API'nin güncellemelerini takip etmek ve çalışmayi güncel tutmak da önemlidir.
    \item Dokümantasyon ve Destek: çalışmaye ilişkin kapsamlı bir dokümantasyon oluşturmak, kullanıcıların çalışmayi daha iyi anlamalarını ve kullanmalarını sağlayacaktır. Ayrıca, kullanıcıların karşılaştıkları sorunlar için destek sağlamak da önemlidir. Sorunların çözümüne yönelik bir destek kanalı veya topluluk oluşturmak, kullanıcı memnuniyetini artıracaktır.
    \item Genişletilebilirlik: çalışmaye ekstra servisler eklemek için Laravel paketlerinin geliştirilmesi teşvik edilmelidir. Geliştiricilere açık API ve belgeleme sağlanarak, yeni servislerin kolayca entegre edilebilmesi ve çalışmanin kullanım alanının genişlemesi sağlanabilir.
\end{itemize}

Sonuç olarak, bu çalışma konteyner teknolojilerinin karmaşıklığını azaltarak kullanım kolaylığı sağlamak amacıyla geliştirilmiştir. Başarılı sonuçları ve önerilerle birlikte, çalışmanin gelecekteki gelişim potansiyeli ve kullanıcı memnuniyetini artırma imkanı vardır.

Bu çalışmanın github reposuna bu linkten ulaşabilirsiniz: \href{https://github.com/ikhalilatteib/these}{https://github.com/ikhalilatteib/these}
Docker uygulaması windows ortamında nasıl çalışır adlı yazıya bu linkten ulaşabilirsiniz: \href{https://medium.com/@celikaleyna71/docker-desktop-ile-konteyner-orkestrasyonu-9a6429637e87}{https://medium.com/@celikaleyna71/docker-desktop-ile-konteyner-orkestrasyonu-9a6429637e87}