\section{SONUÇLAR VE ÖNERİLER}

Bu çalışma, konteyner teknolojilerinin yaygınlaşmasıyla birlikte ortaya çıkan karmaşıklığı azaltmak amacıyla geliştirilmiştir. Konteyner kullanımını kolaylaştırmayı hedefleyen bu çalışmada, konteyner oluşturma, çalıştırma ve sonuçları başarıyla gerçekleştirilmektedir. 

Çalışmanın başarılı sonuçları aşağıdaki şekillerde değerlendirilebilir:
\begin{itemize}
    \item Konteyner İşlemleri: Çalışmada, Docker Engine API ile entegre olarak konteyner oluşturma, çalıştırma, durdurma ve silme gibi işlemleri başarılı bir şekilde gerçekleştirmektedir. Bu sayede kullanıcılar, Docker konteynerlerini kolaylıkla yönetebilmektedir.
    \item Çalışma Performansı: Çalışmanın uygulama performansı, Guzzle kütüphanesinin etkili kullanımı ve Docker Engine API'nin sağladığı hızlı yanıtlar sayesinde yüksek seviyede tutulmaktadır. Bu da kullanıcı deneyimini olumlu yönde etkilemektedir.
    \item Theharvester Servisi: Çalışmada, servislerin eklenmesini desteklemektedir. Örnek olarak Theharvester servisi geliştirilmiş ve başarıyla çalışmaya entegre edilmiştir. Bu sayede kullanıcılar, çalışmalarına farklı servisleri ekleyerek genişletme imkanına sahip olmaktadır.
\end{itemize}




\subsection*{Öneriler}
\begin{itemize}
    \item Kullanıcı Arayüzü Geliştirme: Gelecekte, web tabanlı bir kullanıcı arayüzü oluşturulabilir. Bu arayüz sayesinde kullanıcılar, konteyner oluşturma, silme, güncelleme ve sonuçları inceleme gibi işlemleri görsel bir şekilde gerçekleştirebilir. Bu, kullanım kolaylığı sağlayacak ve çalışmanın kullanıcı tabanını genişletecektir.
    \item Hata Yönetimi ve Güncelleme: Çalışmada, hataların ve istisnaların yönetimini etkin bir şekilde gerçekleştirmektedir. Gelecekte, kullanıcı geri bildirimleri ve testlerle çalışmanın daha da güçlendirilmesi ve hata sayısının en aza indirgenmesi sağlanabilir. Ayrıca, Docker Engine API'nin güncellemelerini takip etmek ve çalışmayı güncel tutmak da önemlidir.
    \item Dokümantasyon ve Destek: Çalışmaya ilişkin kapsamlı bir dokümantasyon oluşturmak, kullanıcıların çalışmayı daha iyi anlamalarını ve kullanmalarını sağlayacaktır. Ayrıca, kullanıcıların karşılaştıkları sorunlar için destek sağlamak da önemlidir. Sorunların çözümüne yönelik bir destek kanalı veya topluluk oluşturmak, kullanıcı memnuniyetini artıracaktır.
    \item Genişletilebilirlik: Çalışmaya ekstra servisler eklemek için Laravel paketlerinin geliştirilmesi teşvik edilmelidir. Geliştiricilere açık API ve belgeleme sağlanarak, yeni servislerin kolayca entegre edilebilmesi ve çalışmanın kullanım alanının genişlemesi sağlanabilir.
\end{itemize}

Sonuç olarak, bu çalışma konteyner teknolojilerinin karmaşıklığını azaltarak kullanım kolaylığı sağlamak amacıyla geliştirilmiştir. Başarılı sonuçları ve önerilerle birlikte, çalışmanın gelecekteki gelişim potansiyeli ve kullanıcı memnuniyetini artırma imkanı vardır.

Bu rapor reposunda, readme.md dosyasında aşağıdaki maddelerin linklerine ulaşabilirsiniz:
\begin{itemize}
    \item Projenin Github reposu
    \item Theharvester servisinin github reposu
    \item Veritabani şeması
    \item Windows ortamında Docker'ın çalıştırılması
\end{itemize}

Bu raporun readme linki:\\ https://github.com/ikhalilatteib/mutli-container-project/blob/master/readme.md