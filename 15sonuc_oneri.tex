\section{SONUÇLAR VE ÖNERİLER}
Docker projesinde büyük çaplı bir projeye ihtiyaç duyulmadığı için Kubernetes veya Swarm gibi Docker'in oluşturduğu yönetim araçlarını kullanmaya gerek yoktur. Docker, projesinde ihtiyaçları karşılamak için yeterli olacaktır. Bu nedenle, projenin ölçeği ve karmaşıklığı dikkate alındığında, Docker'in hafif ve taşınabilir konteynerler sağlaması, izole çalışma ortamı sunması ve uygulama dağıtımını kolaylaştırması avantajlıdır. Bu sayede projedeki gereksinimleri etkili bir şekilde yönetebilir ve geliştirilebilir.\\

Proje büyüdükçe ve veritabanı gereksinimleri karmaşıklaştıkça, MySQL yerine PostgreSQL tercih edilebilir. PostgreSQL, karmaşık veri yapıları ve büyük veri hacimlerini etkili bir şekilde işleyebilme özelliğiyle öne çıkar. Ayrıca veri bütünlüğü ve güvenlik konularında da güçlü bir seçenektir. PostgreSQL'in ölçeklenebilirlik yetenekleri, büyük ölçekli sistemlerde yaygın olarak kullanılmasını sağlar.

