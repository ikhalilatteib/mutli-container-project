\begin{center}
{\bf{\large ÖZET}\vspace*{.5cm}

BİTİRME ÇALIŞMASI

Web Tabanlı Konteyner Orkestrasyon Sistemi

ALEYNA ÇELİK \\ IBRAHIM KHALIL ATTEIB YACOUB}

\begin{singlespace}
{\bfseries
Bilecik Şeyh Edebali Üniversitesi\\
Mühendislik Fakültesi\\
Bilgisayar Mühendisliği Bölümü}
\end{singlespace}

{\bf Danışman: Prof. Dr. Cihan KARAKUZU \\ Dr. Öğr. Üyesi Burakhan ÇUBUKÇU}

{\bf \the\year, \ref{TotPages} Sayfa}

\begin{tabular}{p{7cm}p{2cm}p{4cm}}
\center\textbf{Jüri\; Üyeleri}&&\center\textbf{İmza}\cr
\dotfill&&\dotfill\\
\dotfill&&\dotfill\\
\dotfill&&\dotfill
\end{tabular}
\end{center}
{\small  Docker ile çoklu konteynerlar kullanarak paralel işlem yapan sistemler, uygulama ve servislerin farklı konteynerlar içinde izole edilerek çalıştırılmasını sağlayan bir sistem tasarımıdır. Bu sayede her bir konteyner, farklı özellikler ve işlevler için optimize edilebilir ve uygulama ölçeklenebilir hale gelir.
Ayrıca, verilen işi  belirtilen sayıda konteyner ile bölerek daha esnek hale getirmek amaçlanmıştır. Arayüz, canlı olarak ekranda ilerleme bilgileri göstererek, kullanıcılara toplam iş miktarı ile tamamlanma yüzdesi ve hangi konteynerların en hızlı olduğu gibi önemli bilgileri verir.
Bu arayüz, güvenlik testleri veya izinsiz giriş girişimi gibi meşru olmayan amaçlarla kullanılmamalıdır.}


%%%%%%%%%%%%%%%%%%%%%%%%%%%%%%%%%%%%%%%%%%%%%%%%%%%%%%%%%%%%%%%%%%% 
\newpage
\begin{center}
{\bf{\large ABSTRACT}\vspace*{.5cm}

THESIS

Web Based Container Orchestration System

ALEYNA ÇELİK \\ IBRAHIM KHALIL ATTEIB YACOUB}

\begin{singlespace}
{\bf
Bilecik Şeyh Edebali University\\
Engineering Faculty\\
Department of Computer Engineering}
\end{singlespace}

{\bf Advisor : Prof. Dr. Cihan KARAKUZU \\ Assoc. Prof. Dr. Burakhan ÇUBUKÇU

\the\year, \ref{TotPages} Pages}

\begin{tabular}{p{7cm}p{2cm}p{4cm}}
\center \textbf{Jury}&&\center \textbf{Sign}\cr
\dotfill& &\dotfill\\
\dotfill& &\dotfill\\
\dotfill& &\dotfill
\end{tabular}
\end{center}
{\small Using multiple containers with Docker to perform parallel processing is a system design that allows applications and services to be run in different containers, isolated from each other. This allows each container to be optimized for different features and functions, making the application scalable. Additionally, the system is designed to make the given task more flexible by dividing it into a specified number of containers. The interface displays progress information live on the screen, providing users with important information such as the total amount of work completed, the completion percentage, and which containers are the fastest. However, this interface should not be used for illegitimate purposes such as security testing or unauthorized access attempts. }


