\begin{center}
{\bf{\large ÖZET}\vspace*{.5cm}

BİTİRME ÇALIŞMASI

Web Tabanlı Konteyner Orkestrasyon Sistemi

ALEYNA ÇELİK \\ IBRAHIM KHALIL ATTEIB YACOUB}

\begin{singlespace}
{\bfseries
Bilecik Şeyh Edebali Üniversitesi\\
Mühendislik Fakültesi\\
Bilgisayar Mühendisliği Bölümü}
\end{singlespace}

{\bf Danışman: Prof. Dr. Cihan KARAKUZU \\ Dr. Öğr. Üyesi Burakhan ÇUBUKÇU}

{\bf \the\year, \ref{TotPages} Sayfa}

\begin{tabular}{p{7cm}p{2cm}p{4cm}}
\center\textbf{Jüri\; Üyeleri}&&\center\textbf{İmza}\cr
\dotfill&&\dotfill\\
\dotfill&&\dotfill\\
\dotfill&&\dotfill
\end{tabular}
\end{center}
{\small Bu proje, konteyner teknolojilerini kolaylaştırarak kullanımını geliştirmeyi hedeflemektedir. Docker Engine API'sini kullanarak konteyner oluşturma, yürütme ve yönetme işlemlerini kolaylaştırmaktadır. Projede API'ya başarılı bir şekilde bağlantı sağlanmakta ve API çağrılarını verimli bir şekilde gerçekleştirmek için Guzzle kütüphanesi kullanılmaktadır. Ayrıca, "Theharvester" servisinin projeye dahil edilmesi gibi servis entegrasyonunu da desteklemektedir. Gelecekteki geliştirmeler arasında web tabanlı bir arayüz, geliştirilmiş hata yönetimi, kapsamlı dokümantasyon ve genişletilebilirlik bulunabilir. Genel olarak, bu proje konteyner kullanımını basitleştirmeyi ve üretkenliği artırmayı amaçlamaktadır.}

%%%%%%%%%%%%%%%%%%%%%%%%%%%%%%%%%%%%%%%%%%%%%%%%%%%%%%%%%%%%%%%%%%% 
\newpage
\begin{center}
{\bf{\large ABSTRACT}\vspace*{.5cm}

THESIS

Web Based Container Orchestration System

ALEYNA ÇELİK \\ IBRAHIM KHALIL ATTEIB YACOUB}

\begin{singlespace}
{\bf
Bilecik Şeyh Edebali University\\
Engineering Faculty\\
Department of Computer Engineering}
\end{singlespace}

{\bf Advisor : Prof. Dr. Cihan KARAKUZU \\ Assoc. Prof. Dr. Burakhan ÇUBUKÇU

\the\year, \ref{TotPages} Pages}

\begin{tabular}{p{7cm}p{2cm}p{4cm}}
\center \textbf{Jury}&&\center \textbf{Sign}\cr
\dotfill& &\dotfill\\
\dotfill& &\dotfill\\
\dotfill& &\dotfill
\end{tabular}
\end{center}
{\small This project simplifies container technologies and enhances their usability. Leveraging the Docker Engine API, it enables users to effortlessly create, execute, and manage containers. The project successfully establishes a connection with the API and utilizes the Guzzle library for efficient API calls. It also supports service integration, exemplified by the inclusion of the "Theharvester" service. Future improvements may involve a web-based interface, improved error handling, comprehensive documentation, and enhanced extensibility. Overall, this project aims to streamline container usage and boost productivity.}