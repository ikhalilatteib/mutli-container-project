% kaynak başlığını tanımlar
\renewcommand{\refname}{KAYNAKLAR}

\addcontentsline{toc}{section}{KAYNAKLAR}
\begin{thebibliography}{99} %kaynak ortamı 
\bibitem{k1}\url{Brogi, A., Pahl, C., and Soldani, J. 2020. On enhancing the orchestration of multi-container docker applications. In Advances in Service-Oriented and Cloud Computing: Workshops of ESOCC 2018, Como, Italy, September 12–14, 2018, Revised Selected Papers, volume 7, pages 21–33.Springer International Publishing.} [Ziyaret Tarihi: 30 Mayıs 2022]
\bibitem{k2}\url{Docker 2023. Docker overview.} [Ziyaret Tarihi: 12 Mart 2023]
\bibitem{k3}\url{Ibrahim, M. H., Sayagh, M., and Hassan, A. E. 2021. A study of how docker compose is used to compose multi-component systems. Empirical Software Engineering, 26:1 - 27} [Ziyaret Tarihi: 6 Nisan 2022] 
\bibitem{k4}\url{Sharma, V., Saxena, H. K., and Singh, A. K. (2020). Docker 3 for multi-containers web application. In 2020 2nd International Conference on Innovative Mechanisms for Industry Applications (ICIMIA), pages 589–592. IEEE.} [Ziyaret Tarihi: 6 Nisan 2022] 
\bibitem{k5}\url{McKendrick, R. (2020). Mastering Docker: Enhance your containerization and DevOps skills to deliver production-ready applications. Packt Publishing Ltd.} [Ziyaret Tarihi: 6 Nisan 2022] 
\bibitem{k6}\url{Warrier, A. (2020). Containers vs virtual machines (vms). } [Ziyaret Tarihi: 16 Mart 2023.] 
\end{thebibliography}