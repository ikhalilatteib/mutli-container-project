\section{PROJE TANITIMI }
\subsection{Giriş}
Projemizin amacı, Docker kullanarak paralel işlem yapabilen sistemler, uygulamalar ve servisler için çoklu konteynerlar kullanarak bir sistem tasarlamaktır. Docker, izolasyon sağlayarak her bir konteyneri farklı özellikler ve işlevler için optimize etme imkanı sunar ve bu da uygulamanın ölçeklenebilir olmasını sağlar.
\subsection{Yapı ve Teknolojiler:}
Projede, Laravel yapısı kullanıldı. Bu sayede, arayüz ve backend arasında esneklik sağlayarak birlikte çalışma imkanı elde edildi. Arayüz için başlangıçta bir hazır template kullanıldı, ancak bu templateleri ihtiyaçlarımıza göre özelleştirildi ve değişiklikler yapıldı. Daha sonra backend ile entegrasyonu gerçekleştirerek ayarlamaları tamamlandı.
\subsection{Güvenlik Önlemleri:}
Projenin güvenliği için, login sayfası tasarlandı ve erişim kısıtlamaları eklendi. Bu sayede, giriş yapmayan kullanıcıların diğer sayfalara erişmesi engellendi. Kullanıcıların güvenliğini ve verilerin gizliliğini korumak önemli bir hedefimiz oldu.
\subsection{Docker ile Konteynerleme:}
Son aşamada, Docker işlemlerini gerçekleştirerek konteynerleme işlemleri tamamlandı. Docker'ın avantajlarından yararlanarak, uygulamayı farklı konteynerlar içinde çalıştırılabildi. Böylelikle, her bir konteynerin optimize edilmiş bir şekilde çalışmasını sağlayarak işleri belirtilen sayıda konteynera bölerek daha esnek bir yapı oluşturuldu. Bu da performansı artırırken sistem üzerindeki yükü dengelememize yardımcı oldu.
\subsection{İlerleme Takibi ve Bildirim Sistemi:}
Projenin arayüzünde, canlı olarak ilerleme bilgilerini gösteren bir özellik bulunmaktadır. Kullanıcılara, toplam iş miktarı, tamamlanma yüzdesi ve en hızlı çalışan konteynerlar gibi önemli bilgiler sunarak işlerin durumunu takip etmeleri sağlandı. Bu sayede kullanıcılar, işlerin ne kadar ilerlediğini ve hangi konteynerların en etkili olduğunu görebilmektedir.
\subsection{Sonuç:}
Özetlemek gerekirse, projemizde Docker kullanarak paralel işlem yapabilen bir sistem tasarlandı. Laravel yapısını kullanarak esnek bir yapı oluşturuldu. Güvenlik önlemlerini alarak kullanıcıların ve verilerin güvenliğini sağlandı. Docker ile konteynerleme işlemlerini gerçekleştirerek
uygulamayı optimize ettik ve performansı artırdık. Arayüzde canlı ilerleme bilgilerini gösteren bir özellik ekleyerek kullanıcıların işlerin durumunu takip etmelerini kolaylaştırdık.

Projenin başarıyla tamamlanması için bir dizi adım takip edildi. İlk olarak, gereksinimlerimizi belirledik ve Docker'ı seçtik çünkü konteynerleme teknolojisi olarak bize izolasyon ve esneklik sağlayan bir çözüm sunuyordu. Ardından, Laravel yapısını kullanarak arayüz ve backend arasında etkili bir iletişim ve veri akışı sağladık.

Arayüz tarafında, hazır bir template kullanarak başladık, ancak bunu kendi tasarım ve işlevsel gereksinimlerimize uygun şekilde özelleştirdik. Kullanıcılarımızın deneyimini en üst düzeye çıkarmak için kullanıcı dostu ve sezgisel bir arayüz tasarladık.

Backend tarafında, güvenlik önlemleri almak için login sayfasını tasarladık ve kullanıcıların kimlik doğrulama sürecini sağlamlaştırdık. Bu sayede, yalnızca yetkilendirilmiş kullanıcıların sistemimizi kullanabilmesini sağladık ve veri güvenliğini sağladık.

Docker kullanarak konteynerleme işlemlerini gerçekleştirdik. Her bir konteyner, farklı özelliklere ve işlevlere sahip olduğu için optimize edilebilir hale geldi. İşleri parçalara bölerek belirli sayıda konteynera dağıttık, bu da sistemimizin esnekliğini artırdı ve işleri daha hızlı ve etkin bir şekilde gerçekleştirebildi.

Projenin devamında, kullanıcıların işleri daha etkili bir şekilde yönetmelerini sağlamak için bazı ek özellikler ekledik. Örneğin, kullanıcılar işleri önceliklendirebilir ve farklı konteynerlarda çalışacak işlere öncelik atayabilir. Bu, daha hızlı tamamlanması gereken işlerin öncelikli olarak çalıştırılmasını sağladı ve genel performansı artırdı.

Ayrıca, proje takımının iletişimini kolaylaştırmak için bir bildirim sistemi entegre ettik. Kullanıcılar, tamamlanan işler hakkında anlık bildirimler aldı ve gerektiğinde müdahale edebildi. Bu, ekip üyelerinin işlerin durumunu takip etmesini ve projenin ilerlemesini koordine etmesini kolaylaştırdı.

Sonuç olarak, Docker kullanarak paralel işlem yapabilen bir sistem tasarladık ve bu sistemde Laravel yapısını kullanarak esneklik sağladık. Güvenlik önlemleri alarak kullanıcıların ve verilerin güvenliğini koruduk. Docker ile konteynerleme işlemlerini gerçekleştirerek optimize edilmiş bir sistem oluşturduk. Arayüzde ilerleme takibi ve bildirim sistemi gibi özellikler ekleyerek kullanıcıların işleri daha etkili bir şekilde yönetmelerini sağladık.

Bu proje sayesinde, paralel işlem yapabilen sistemlerin performansını artırma, esneklik sağlama ve güvenliği sağlama konularında önemli bir deneyim kazandık.
\section{KULLANILAN YAZILIMLAR VE YÖNTEMLER}
\subsection{Neden Tercih Ettik ?}
\textbf{Laravel}, web uygulamaları için tercih ettiğimiz bir framework'tür. MVC yapısı ve geliştirme kolaylığı sağlar. Hazır bileşenleriyle hızlı ve verimli bir şekilde uygulama geliştirebiliriz. Geniş topluluğuyla destek alabileceğimiz bir kaynağa sahiptir. Bu nedenlerle, Laravel'i tercih ettik.\\
\textbf{Docker'ı} tercih etmemizin nedeni, uygulamaların izole edilmiş, taşınabilir ve hızlı bir şekilde dağıtılmasını sağlayan konteyner teknolojisiyle geliştirme ve dağıtım süreçlerini kolaylaştırmasıdır.\\
\textbf{MySQLi}, MySQL veritabanı ile PHP arasında bir bağlantı sağlayan bir PHP eklentisidir. MySQLi'nin tercih edilmesinin nedeni, gelişmiş özellikleri, performans iyileştirmeleri ve güvenlik önlemleriyle MySQL ile daha etkili ve güvenli bir şekilde etkileşim sağlamasını sağlamasıdır.
\subsection{Web Uygulama Geliştirmek İçin \textbf{Laravel}}
Laravel, popüler bir PHP framework'üdür ve MVC (Model-View-Controller) mimarisini kullanır. Bu mimari, uygulamanın farklı katmanlarını ayırarak kodun düzenli, okunabilir ve bakımı kolay hale gelmesini sağlar. Laravel, model, view, controller ve route gibi temel bileşenlerden oluşur. Model, veri katmanını temsil ederken, view kullanıcı arayüzünü ve controller iş mantığını yönetir. Route ise istekleri doğru controller'a yönlendirir.\\

Laravel'in MVC yapısı, kodun düzenli tutulmasını, bileşenlerin bağımsız olmasını ve geliştirme sürecinin daha etkili olmasını sağlar. Bu sayede uygulama daha yeniden kullanılabilir, test edilebilir ve bakımı kolay hale gelir.
\subsubsection{Kullanım Kolaylığı:}
Laravel, kullanıcı dostu ve kolay bir kullanım deneyimi sunar. Gelişmiş komut satırı araçları, hazır bileşenler ve entegre özellikler sayesinde hızlı bir şekilde uygulama geliştirebilirsiniz. Laravel'in sunduğu Artisan komut satırı aracı, geliştirme sürecini hızlandırır ve tekrarlayan görevleri otomatikleştirir. Artisan komutları, veritabanı yönetimi, migrasyonlar, testlerin çalıştırılması, güvenlik kontrolleri ve daha birçok işlemi kolaylıkla gerçekleştirmenizi sağlar.
Ayrıca, Laravel'in içerdiği hazır bileşenler ve entegre özellikler, geliştirme sürecini daha da kolaylaştırır. Örneğin, kimlik doğrulama ve yetkilendirme sistemleri gibi önemli işlevlerin birçoğu Laravel'de hazır olarak sunulur ve uygulamanızı güvenli hale getirmek için kolayca kullanılabilir. Ayrıca, oturum yönetimi, önbellekleme, e-posta gönderme, dosya depolama ve diğer yaygın kullanılan işlevler için Laravel'in entegre özelliklerinden yararlanabilirsiniz.
\subsubsection{MVC Mimari ve Yapılandırma: }
Laravel'in Model-View-Controller (MVC) mimarisi, uygulamanın farklı katmanlarını ayırarak kodun düzenli tutulmasını ve bakımını kolaylaştırır. Model, view ve controller bileşenleri arasındaki net ayrım, iş mantığının yönetimini ve kullanıcı arayüzünün ayrı tutulmasını sağlar.
\subsubsection{Etkili Veritabanı Yönetimi: }Laravel, veritabanı işlemlerini kolaylaştıran bir ORM (Object-Relational Mapping) olan Eloquent'i içerir. Eloquent, veritabanı sorgularını nesne odaklı bir şekilde yapmanızı sağlar ve veritabanı işlemlerini hızlı ve verimli bir şekilde gerçekleştirmenizi sağlar.
\subsubsection{Veritabanı Şeması:} Laravel'in sağladığı göç (migration) sistemini kullanarak, veritabanı şemasını kolayca oluşturabilir ve değişiklikleri yönetebilirsiniz. Bu, uygulamanızın veritabanı yapısını güncellemek ve yönetmek için esneklik sağlar.
\subsubsection{Güvenlik ve Oturum Yönetimi:}
Laravel, güvenlik önlemleriyle donatılmıştır. Oturum yönetimi, kimlik doğrulama ve yetkilendirme gibi güvenlik işlemlerini kolayca uygulamanızı sağlar.
\subsubsection{Geniş Ekosistem ve Topluluk Desteği: }
Laravel, büyük bir geliştirici topluluğuna sahiptir ve aktif bir ekosistemi vardır. Laravel Forge, Laravel Nova, Laravel Horizon gibi araçlar ve paketlerle işleri kolaylaştırabilir ve projenizi hızlandırabilirsiniz. Ayrıca, Laravel ile ilgili belgeler, kaynaklar ve topluluk destekleri kolayca erişilebilir.
\subsection{Konteynerize Etmek İçin \textbf{Docker }}
\subsubsection{Docker Nedir?}

Docker, uygulamaların ve hizmetlerin paketlenmesi, dağıtımı ve çalıştırılmasını kolaylaştıran bir yazılım platformudur. Docker, konteyner teknolojisi kullanarak uygulamaların izole edilmiş ve taşınabilir bir şekilde çalışmasını sağlar.

\subsubsection{Docker'ın çalışma yapısı }
Docker'ın çalışma yapısı şu şekildedir:
\begin{itemize}
\item Konteyner: Docker'ın temel yapı birimidir. Konteyner, uygulamanın tüm bağımlılıklarını (kod, çalışma zamanı, kütüphaneler, ortam değişkenleri vb.) bir araya getirir ve izole bir ortamda çalışmasını sağlar. Konteynerlar, bir uygulamayı çalıştırmak için gerekli olan tüm bileşenleri içerir ve bu sayede uygulamaları farklı ortamlarda tutarlı bir şekilde çalıştırabilirsiniz.

\item Docker Image (Docker İmajı): Bir Docker konteynerini oluşturmak için kullanılan bir şablondur. Bir Docker imajı, bir veya daha fazla katman (layer) olarak adlandırılan yapı taşlarından oluşur. Her katman, imajın farklı bir bileşenini veya yapılandırmasını temsil eder. Bir Docker imajı, bir veya birden fazla konteyneri oluşturmak için kullanılabilir.

\item Dockerfile (Docker Dosyası): Docker imajlarının nasıl oluşturulacağını tanımlayan bir metin dosyasıdır. Dockerfile, bir uygulamanın çalıştırılması için gerekli olan adımları ve komutları belirtir. Dockerfile, uygulama kodunu, çalışma zamanını, bağımlılıkları ve diğer yapılandırmaları imaja dahil etmek için kullanılır.

\item Docker Registry: Docker imajlarının depolandığı ve paylaşıldığı bir merkezi kaynak. Docker Hub, en popüler ve yaygın kullanılan Docker Registry'dir. Docker Hub'da, birçok hazır Docker imajı bulabilir ve kendi imajlarınızı da paylaşabilirsiniz. Ayrıca, Docker Registry'lerini yerel olarak da kurabilir ve kullanabilirsiniz.
\end{itemize}

\subsubsection{Docker'ın avantajları }
Docker'ın avantajları şunlardır:
\begin{itemize}
\item Taşınabilirlik: Docker, uygulamaların farklı platformlarda ve ortamlarda sorunsuz bir şekilde çalışmasını sağlar. Bir kez oluşturulan Docker imajı, farklı sistemlerde kolayca dağıtılabilir ve çalıştırılabilir.

\item İzolasyon: Docker konteynerleri, uygulamaları ve bağımlılıklarını izole ederek, bir konteynerin diğerlerine etkileşimde bulunmasını engeller. Bu, güvenlik ve istikrar açısından önemli bir avantaj sağlar.

\item Hızlı dağıtım ve ölçeklendirme: Docker, uygulamaların hızlı bir şekilde dağıtılmasını ve ölçeklendirilmesini sağlar. İmajların hızlı bir şekilde oluşturulması ve konteynerlerin hızlı bir şekilde başlatılması sayesinde uygulamaların hızlıca yayınlanması mümkün olur.

\item Verimlilik: Docker, kaynakların etkin bir şekilde kullanılmasını sağlar. Konteynerler, daha az bellek ve işlemci gücü tüketerek daha fazla uygulamanın çalışmasını sağlar.

\item Veri bütünlüğü: Docker, verilerin konteynerlerle birlikte taşınmasını ve yönetilmesini kolaylaştırır. Konteynerler, verileri izole bir şekilde tutar ve bu da veri bütünlüğünü sağlar.

\item Hızlı geri alma ve güncelleme: Docker, imaj ve konteyner tabanlı bir yaklaşım kullanarak hızlı geri alma ve güncelleme süreçleri sağlar. Eski bir imaja geri dönerek veya güncellenmiş bir imajla yeni bir konteyner oluşturarak sistemde yapılan değişiklikleri geri almak veya güncellemek kolaydır.

\item Kaynak verimliliği: Docker, paylaşılan çekirdek özelliklerini kullanarak kaynakların daha verimli bir şekilde kullanılmasını sağlar. Birçok konteyner, tek bir işletim sistemi çekirdeği üzerinde çalışır ve bu da sistem kaynaklarının daha etkin bir şekilde kullanılmasını sağlar.

\item Topluluk desteği: Docker, geniş bir kullanıcı ve geliştirici topluluğuna sahiptir. Bu topluluk, Docker hakkında bilgi paylaşımı, sorun giderme ve yenilikçi çözümler sunma konusunda yardımcı olur. Docker Hub gibi kaynaklar, hazır kullanıma uygun imajlar ve araçlar sağlar.
\end{itemize}

Docker'ın bu avantajları, uygulama geliştirme ve dağıtım süreçlerinde hızlılık, esneklik, güvenlik ve ölçeklenebilirlik sağlar. Ayrıca, Docker'ın genişletilebilirlik ve entegrasyon yetenekleri, farklı platformlar ve araçlarla uyumlu çalışmayı kolaylaştırır.

\subsubsection{Neden Docker ?}
Docker, bir yazılımın uygulama ve bağımlılıklarını bir "konteyner" olarak paketlemek ve çalıştırmak için kullanılan bir platformdur. Docker'ın sanal makinelerden farklı olarak konteynerizasyon teknolojisi üzerine kurulu olması, birkaç önemli avantaj sunmaktadır:\\
\begin{itemize}
\item Daha hafif ve hızlı: Sanal makineler (VM'ler), her biri kendi işletim sistemini çalıştıran ayrı bir sanal makine olduğundan, daha fazla kaynak tüketir ve yavaş çalışabilir. Docker konteynerleri ise ana işletim sistemini paylaşır ve sadece uygulama ve bağımlılıkları içerir. Bu nedenle, Docker konteynerleri daha hafif ve daha hızlıdır.\\

\item Daha verimli kaynak kullanımı: Docker, aynı fiziksel makine üzerinde birden çok konteyner çalıştırabilme yeteneği sayesinde kaynakları daha verimli kullanmanızı sağlar. Sanal makinelerde her bir VM, kendi işletim sistemini çalıştırdığından kaynak israfı yaşanabilirken, Docker konteynerleri aynı işletim sistemini paylaşarak daha etkili bir kaynak yönetimi sağlar.\\

\item Taşınabilirlik ve uyumluluk: Docker konteynerleri, uygulama ve bağımlılıklarını bir araya getirdiği için taşınabilirlik ve uyumluluk sağlar. Konteynerler, herhangi bir ortamda, herhangi bir makinede aynı şekilde çalışabilir. Bir kez oluşturulduklarında, Docker konteynerleri kolayca dağıtılabilir ve başka bir makineye taşınabilir.\\

\item Hızlı dağıtım ve ölçeklendirme: Docker konteynerleri, uygulamaların hızlı dağıtımını ve ölçeklendirilmesini sağlar. Konteynerlerin hızlı bir şekilde başlatılması ve durdurulması mümkündür. Ayrıca, bir uygulamayı birden çok konteyner olarak çalıştırarak yüksek kullanılabilirlik ve ölçeklenebilirlik elde etmek kolaydır.\\

\item İzolasyon ve güvenlik: Docker konteynerleri, birbirinden izole edilmiş çalışma ortamları sağlar. Her konteyner, kendi dosya sistemini, ağ bağlantılarını ve süreçlerini izole eder. Bu, bir konteynerin diğerlerinden etkilenmeyeceği anlamına gelir, böylece daha güvenli bir ortam sağlar.\\
\end{itemize}
Docker komutlarını ve açıklamalarını içeren Tablo \ref{tab:docker-komutlar}.'da görebilirsiniz.
\begin{table}[h]
  \centering
  \caption{Docker Komutları ve Açıklamaları}
  \label{tab:docker-komutlar}
  \begin{tabular}{|p{0.3\textwidth}|p{0.6\textwidth}|}
  \hline
  \textbf{Komut} & \textbf{Açıklama} \\ \hline
  docker run & Bir konteynerin başlatılması ve çalıştırılması için kullanılır. \\ \hline
  docker stop & Çalışan bir konteyneri durdurmak için kullanılır. \\ \hline
  docker build & Bir Docker imajının oluşturulması için kullanılır. \\ \hline
  docker push & Bir Docker imajının uzak bir imaj deposuna yüklenmesi için kullanılır. \\ \hline
  docker pull & Bir Docker imajının uzak bir imaj deposundan indirilmesi için kullanılır. \\ \hline
  docker exec & Çalışan bir konteynere komut çalıştırmak için kullanılır. \\ \hline
  docker ps & Çalışan konteynerleri listelemek için kullanılır. \\ \hline
  docker images & Oluşturulmuş Docker imajlarını listelemek için kullanılır. \\ \hline
  \end{tabular}
\end{table}
\subsubsection{Docker vs Sanal Makianlar}
Sanal makine (VM) teknolojisi, ayrı işletim sistemleri ve bağımsız çalışma ortamları sağlar. Her bir sanal makine, kendi işletim sistemi, yazılım yığını ve kaynaklara sahip bir sanal bilgisayardır. VM'ler, fiziksel makine üzerinde çalışırken, her biri kendi kaynaklara sahip olduğundan izolasyon sağlarlar. Bu, farklı işletim sistemlerini aynı fiziksel makinede çalıştırmanın bir yolunu sunar. Ancak, VM'lerin daha fazla bellek, işlemci gücü ve depolama alanı gibi kaynaklara ihtiyaç duyması ve daha yavaş çalışması dezavantajlarından bazılarıdır.\\
\begin{figure}[]
    \centering
    \includegraphics[width=0.9\textwidth]{images/virtualboxvsdocker.jpg}
    \caption{Docker vs Sanal Makina}
    \label{fig:resim_etiketi}
  \end{figure}
Docker konteynerleri, hafif ve taşınabilir uygulama çalıştırma birimleridir. Docker, aynı çekirdek üzerinde çalışan ve paylaşılan kaynakları kullanan konteynerleştirilmiş uygulamaları yönetmek için bir platform sağlar. Her bir Docker konteyneri, izole bir çalışma ortamı sunar, ancak farklı bir işletim sistemi çalıştırmak yerine, ana makinedeki işletim sistemi ve çekirdek üzerinde çalışır. Bu, daha düşük bellek kullanımı ve daha hızlı başlatma süreleri gibi avantajlar sağlar. Ayrıca, Docker konteynerleri uygulama dağıtımını kolaylaştırır ve yazılımın farklı ortamlarda tutarlı bir şekilde çalışmasını sağlar.\\
Biz sanal makina olarak VirtualBox'ı baz alacağız.\\
Tablo \ref{tab:docker-virtualbox} Docker ile VirtualBox arasındaki karşılaştırmayı göstermektedir.\\
\subsubsection{Docker vs VirtualBox}
\begin{table}[!h]
  \centering
  \begin{tabular}{|p{0.45\textwidth}|p{0.45\textwidth}|}
    \hline
    \textbf{Docker} & \textbf{VirtualBox} \\
    \hline
    Konteyner tabanlı sanallaştırma & Tam sanallaştırma \\
    \hline
    Aynı çekirdek üzerinde çalışır & Farklı işletim sistemlerini çalıştırır \\
    \hline
    Hafif ve taşınabilir konteynerler & Tam işletim sistemine sahip sanal makineler \\
    \hline
    İzole çalışma ortamı sağlar & İzole çalışma ortamı sağlar \\
    \hline
    Daha düşük bellek kullanımı ve hızlı başlatma & Daha fazla kaynak gerektirir \\
    \hline
    Uygulama dağıtımını kolaylaştırır & Farklı işletim sistemlerinin testi için kullanılır \\
    \hline
    Docker Hub'da konteyner görüntüleri paylaşabilir & Sanal makine görüntüleri kullanılabilir \\
    \hline
    Microservisler ve ölçeklenebilir uygulama geliştirmeleri & Özel izolasyon gerektiren durumlar \\
    \hline
  \end{tabular}
  \caption{Docker vs. VirtualBox}
  \label{tab:docker-virtualbox}
\end{table}
\subsection{Veri Tabanı İçin \textbf{MYSQL }}
\subsubsection{MYSQL Nedir?}
MySQLi, PHP ile MySQL veritabanına erişmek için kullanılan gelişmiş bir PHP eklentisidir. MySQLi, MySQL eklentisine göre daha fazla özellik ve performans sağlar. Bu eklenti, PHP'nin dahili fonksiyonları ve sınıfları aracılığıyla MySQL sunucusuna bağlanma, sorgu yürütme, veri ekleme, güncelleme, silme gibi veritabanı işlemlerini kolaylıkla gerçekleştirme imkanı sunar. MySQLi'nin nesne yönelimli programlama (OOP) sınıfları, parametreli sorguların kullanımı, hata yönetimi ve bağlantı kontrolü gibi gelişmiş özellikleri sayesinde veritabanı işlemleri daha güvenli, verimli ve esnek hale gelir. Bu nedenle, MySQLi, PHP projelerinde MySQL veritabanına erişmek için tercih edilen bir seçenektir.

MySQL, açık kaynaklı bir ilişkisel veritabanı yönetim sistemidir ve verileri tablolarda saklayarak ilişkileri yönetmeyi sağlar. İstemci-sunucu modeline dayanan bu çok kullanıcılı veritabanı sistemi, bir sunucu üzerinde çalışır. MySQL, hızlı, güvenilir ve geniş bir kullanıcı tabanına sahiptir. Ayrıca, geniş uyumluluk listesi sayesinde farklı programlama dilleri ve platformlarla kolayca entegre olabilir. Veritabanı yönetimi, veri manipülasyonu, sorgulama, güvenlik, yedekleme ve geri yükleme gibi birçok temel işlemi destekler. Bu nedenlerle MySQL, web uygulamalarında ve büyük ölçekli sistemlerde sıkça tercih edilen bir veritabanı çözümüdür. Yüksek performans, ölçeklenebilirlik ve kullanıcı dostu arayüz gibi önemli özelliklere sahiptir, veri tabanı yönetimi için güçlü ve esnek bir seçenektir.
\subsubsection{Özellikler ve Avantajlar:}
MySQL, geniş kullanıcı tabanıyla tanınan hızlı ve güvenilir bir veritabanı yönetim sistemidir. Veri manipülasyonu, sorgulama, güvenlik, yedekleme ve geri yükleme gibi temel veritabanı işlemlerini başarılı bir şekilde destekler. Ayrıca, MySQL geniş bir uyumluluk listesine sahiptir ve farklı programlama dilleri ve platformlarla kolayca entegre olabilir. Veri bütünlüğünü koruma, güvenlik, yüksek performans, ölçeklenebilirlik ve kullanıcı dostu bir arayüz gibi önemli özellikleriyle öne çıkar. Bu özellikleri sayesinde MySQL, veritabanı yönetimi için güçlü bir seçenek olarak tercih edilmektedir.

\subsubsection{Kullanım Alanları:}
MySQL, geniş ölçekteki web uygulamaları ve büyük sistemlerde yaygın olarak kullanılan etkili bir veritabanı çözümüdür. Verilerin etkin bir şekilde saklanmasını ve yönetilmesini sağlayarak projelerin gereksinimlerini karşılar. Güçlü ve esnek yapısıyla veritabanı yönetimi için tercih edilen MySQL, veri tabanlarının güvenli ve verimli bir şekilde işlenmesini sağlar.
\subsubsection{MYSQL vs PostegreSQL}
\begin{table}[!h]
    \centering
    \begin{tabular}{|p{0.45\linewidth}|p{0.45\linewidth}|}
      \hline
      \textbf{MySQL} & \textbf{PostgreSQL} \\ \hline
      Açık kaynaklı bir ilişkisel veritabanı yönetim sistemi. & Açık kaynaklı bir ilişkisel veritabanı yönetim sistemi. \\ \hline
      Hızlı performans ve geniş bir kullanıcı tabanına sahip. & Gelişmiş özellikleri ve veri bütünlüğü konusunda güçlü destek. \\ \hline
      Ölçeklenebilir ve büyük ölçekli sistemlerde yaygın olarak kullanılır. & Veri bütünlüğünü koruma ve güvenlik açısından güçlü bir seçenektir. \\ \hline
      Kolayca farklı platformlar ve programlama dilleriyle entegre olabilir. & Yüksek performans ve sorgu optimizasyonu sunar. \\ \hline
      Düşük sistem gereksinimleri ve hafif bir yapıya sahip. & Tutarlılık ve bütünlük sağlayan bir transaksiyon yönetim sistemi sunar. \\ \hline
      \end{tabular}
    \caption{MySQL ve PostgreSQL Karşılaştırması}
    \label{tab:mysql-postgresql}
\end{table}