\section{PROJE TANITIMI }
Projemizin amacı, Docker kullanarak paralel işlem yapabilen sistemler, uygulamalar ve servisler için çoklu konteynerlar kullanarak bir sistem tasarlamaktır. Docker, izolasyon sağlayarak her bir konteyneri farklı özellikler ve işlevler için optimize etme imkanı sunar ve bu da uygulamanın ölçeklenebilir olmasını sağlar.\\

Projede, Laravel yapısı kullanılıdI. Bu sayede, arayüz ve backend arasında esneklik sağlayarak birlikte çalışma imkanı elde edildi. Arayüz için başlangıçta bir hazır template kullanıldı, ancak bu templateleri ihtiyaçlarımıza göre özelleştirildi ve değişiklikler yapıldı. Daha sonra backend ile entegrasyonu gerçekleştirerek ayarlamaları tamamlandı.\\

Projenin güvenliği için, login sayfasını tasarlandı ve erişim kısıtlamaları eklendi. Bu sayede, giriş yapmayan kullanıcıların diğer sayfalara erişmesi engellendi. Kullanıcıların güvenliğini ve verilerin gizliliğini korumak önemli bir hedefimiz oldu.\\
Son aşamada, Docker işlemlerini gerçekleştirerek konteynerleme işlemleri tamamlandı. Docker'ın avantajlarından yararlanarak, uygulamayı farklı konteynerlar içinde çalıştırılabildi. Böylelikle, her bir konteynerin optimize edilmiş bir şekilde çalışmasını sağlayarak işleri belirtilen sayıda konteynera bölerek daha esnek bir yapı oluşturuldu. Bu da performansı artırırken sistem üzerindeki yükü dengelememize yardımcı oldu.\\
Projenin arayüzünde, canlı olarak ilerleme bilgilerini gösteren bir özellik bulunmaktadır. Kullanıcılara, toplam iş miktarı, tamamlanma yüzdesi ve en hızlı çalışan konteynerlar gibi önemli bilgiler sunarak işlerin durumunu takip etmeleri sağlandı. Bu sayede kullanıcılar, işlerin ne kadar ilerlediğini ve hangi konteynerların en etkili olduğunu görebilmektedir.\\
Özetlemek gerekirse, projemizde Docker kullanarak paralel işlem yapabilen bir sistem tasarlandı. Laravel  yapısını kullanarak esnek bir yapı oluşturuldu. Güvenlik önlemlerini alarak kullanıcıların ve verilerin güvenliğini sağlandı. Docker ile konteynerleme işlemlerini gerçekleştirerek uygulamayı optimize ettik ve performansı artırıldı. Arayüzde canlı ilerleme bilgilerini gösteren bir özellik ekleyerek kullanıcıların işlerin durumunu takip etmelerini kolaylaştırıldı.\\
Projemizin başarıyla tamamlanması için bir dizi adım takip edildi. İlk olarak, gereksinimlerimizi belirlendi ve Docker'ı seçtik çünkü konteynerleme teknolojisi olarak bize izolasyon ve esneklik sağlayan bir çözüm sunuyordu. Ardından, Laravel  yapısını kullanarak arayüz ve backend arasında etkili bir iletişim ve veri akışı sağlandı.\\
Arayüz tarafında, hazır bir template kullanarak başlandı, ancak bunu kendi tasarım ve işlevsel gereksinimlerimize uygun şekilde özelleştirildi. Kullanıcılarımızın deneyimini en üst düzeye çıkarmak için kullanıcı dostu ve sezgisel bir arayüz tasarlandı.\\
Backend tarafında, güvenlik önlemleri almak için login sayfasını tasarlandı ve kullanıcıların kimlik doğrulama sürecini sağlamlaştırıldı. Bu sayede, yalnızca yetkilendirilmiş kullanıcıların sistemimizi kullanabilmesi sağlandı ve veri güvenliğini sağlandı.\\
Docker kullanarak konteynerleme işlemleri gerçekleştirildi. Her bir konteyner, farklı özelliklere ve işlevlere sahip olduğu için optimize edilebilir hale geldi. İşleri parçalara bölerek belirli sayıda konteynera dağıttık, bu da sistemimizin esnekliğini artırdı ve işleri daha hızlı ve etkin bir şekilde gerçekleştirebildi.\\
Projenin devamında, kullanıcıların işleri daha etkili bir şekilde yönetmelerini sağlamak için bazı ek özellikler eklendi. Örneğin, kullanıcılar işleri önceliklendirebilir ve farklı konteynerlarda çalışacak işlere öncelik atayabildi. Bu, daha hızlı tamamlanması gereken işlerin öncelikli olarak çalıştırılmasını sağladı ve genel performansı artırdı.\\
Ayrıca, proje takımının iletişimini kolaylaştırmak için bir bildirim sistemi entegre edildi. Kullanıcılar, tamamlanan işler hakkında anlık bildirimler aldı ve gerektiğinde müdahale edebildi. Bu, ekip üyelerinin işlerin durumunu takip etmesini ve projenin ilerlemesini koordine etmesini kolaylaştırdı.\\
Projenin güvenliği için, izinsiz erişim girişimlerini engellemek için oturum açma sayfasına ek güvenlik önlemleri eklendi. Kullanıcıların güvenliği ve verilerin korunması en üst düzeyde önem taşırken, sistemimize giriş yapmak isteyen herhangi bir yetkisiz kişinin engellenmesi sağlandı.\\
Performansı ve ölçeklenebilirliği daha da artırmak için, proje üzerinde sürekli iyileştirmeler ve optimizasyonlar gerçekleştirildi. Örneğin, konteynerler arasında yük dengelemesi yaparak işlerin daha dengeli bir şekilde dağıtılmasını sağlandı ve böylece sistem üzerindeki aşırı yüklenmeyi önlenildi.\\
Arayüzde canlı ilerleme bilgilerini gösteren bir özellik eklendi. Kullanıcılara toplam iş miktarını, tamamlanma yüzdesini ve en hızlı çalışan konteynerları göstererek işlerin durumunu takip etmelerini sağlandı. Bu, kullanıcıların proje ilerlemesini daha iyi yönetmelerini ve gerekirse müdahalede bulunmalarını sağlayan değerli bir geribildirim mekanizması sağlandı.\\
Son olarak, proje yönetimi için bir kontrol paneli eklendi. Bu panel, proje yöneticilerinin işlerin ilerlemesini takip etmelerini, konteyner performansını analiz etmelerini ve gerektiğinde müdahale etmelerini sağlandı. Kontrol paneli, genel bir bakış açısı sunarak proje yönetiminin daha etkin bir şekilde yapılmasına yardımcı oldu.\\
Tüm bu adımları takip ederek, Docker kullanarak paralel işlem yapabilen bir sistem tasarlandı. Projemiz, Laravel  yapısıyla esneklik sağladı, güvenliği ön planda tuttu, konteynerleme ile performansı artırdı ve kullanıcılarımıza canlı ilerleme bilgileri sunarak daha iyi bir deneyim sağladı.\\


\section{KULLANILAN YAZILIMLAR VE YÖNTEMLER}

\subsection{Web Uygulama Geliştirmek İçin}
\subsubsection{Laravel }
Laravel, popüler bir PHP framework'üdür ve MVC (Model-View-Controller) mimarisini kullanır. Bu mimari, uygulamanın farklı katmanlarını ayırarak kodun düzenli, okunabilir ve bakımı kolay hale gelmesini sağlar. Laravel, model, view, controller ve route gibi temel bileşenlerden oluşur. Model, veri katmanını temsil ederken, view kullanıcı arayüzünü ve controller iş mantığını yönetir. Route ise istekleri doğru controller'a yönlendirir.\\

Laravel'in MVC yapısı, kodun düzenli tutulmasını, bileşenlerin bağımsız olmasını ve geliştirme sürecinin daha etkili olmasını sağlar. Bu sayede uygulama daha yeniden kullanılabilir, test edilebilir ve bakımı kolay hale gelir.

\subsection{Konteynerize Etmek İçin  }
\subsubsection{Docker}
Docker, uygulamaların ve hizmetlerin paketlenmesi, dağıtımı ve çalıştırılmasını kolaylaştıran bir yazılım platformudur. Docker, konteyner teknolojisi kullanarak uygulamaların izole edilmiş ve taşınabilir bir şekilde çalışmasını sağlar.

Docker'ın çalışma yapısı şu şekildedir:
\begin{itemize}
\item Konteyner: Docker'ın temel yapı birimidir. Konteyner, uygulamanın tüm bağımlılıklarını (kod, çalışma zamanı, kütüphaneler, ortam değişkenleri vb.) bir araya getirir ve izole bir ortamda çalışmasını sağlar. Konteynerlar, bir uygulamayı çalıştırmak için gerekli olan tüm bileşenleri içerir ve bu sayede uygulamaları farklı ortamlarda tutarlı bir şekilde çalıştırabilirsiniz.

\item Docker Image (Docker İmajı): Bir Docker konteynerini oluşturmak için kullanılan bir şablondur. Bir Docker imajı, bir veya daha fazla katman (layer) olarak adlandırılan yapı taşlarından oluşur. Her katman, imajın farklı bir bileşenini veya yapılandırmasını temsil eder. Bir Docker imajı, bir veya birden fazla konteyneri oluşturmak için kullanılabilir.

\item Dockerfile (Docker Dosyası): Docker imajlarının nasıl oluşturulacağını tanımlayan bir metin dosyasıdır. Dockerfile, bir uygulamanın çalıştırılması için gerekli olan adımları ve komutları belirtir. Dockerfile, uygulama kodunu, çalışma zamanını, bağımlılıkları ve diğer yapılandırmaları imaja dahil etmek için kullanılır.

\item Docker Registry: Docker imajlarının depolandığı ve paylaşıldığı bir merkezi kaynak. Docker Hub, en popüler ve yaygın kullanılan Docker Registry'dir. Docker Hub'da, birçok hazır Docker imajı bulabilir ve kendi imajlarınızı da paylaşabilirsiniz. Ayrıca, Docker Registry'lerini yerel olarak da kurabilir ve kullanabilirsiniz.
\end{itemize}


Docker'ın avantajları şunlardır:
\begin{itemize}
\item Taşınabilirlik: Docker, uygulamaların farklı platformlarda ve ortamlarda sorunsuz bir şekilde çalışmasını sağlar. Bir kez oluşturulan Docker imajı, farklı sistemlerde kolayca dağıtılabilir ve çalıştırılabilir.

\item İzolasyon: Docker konteynerleri, uygulamaları ve bağımlılıklarını izole ederek, bir konteynerin diğerlerine etkileşimde bulunmasını engeller. Bu, güvenlik ve istikrar açısından önemli bir avantaj sağlar.

\item Hızlı dağıtım ve ölçeklendirme: Docker, uygulamaların hızlı bir şekilde dağıtılmasını ve ölçeklendirilmesini sağlar. İmajların hızlı bir şekilde oluşturulması ve konteynerlerin hızlı bir şekilde başlatılması sayesinde uygulamaların hızlıca yayınlanması mümkün olur.

\item Verimlilik: Docker, kaynakların etkin bir şekilde kullanılmasını sağlar. Konteynerler, daha az bellek ve işlemci gücü tüketerek daha fazla uygulamanın çalışmasını sağlar.

\item Veri bütünlüğü: Docker, verilerin konteynerlerle birlikte taşınmasını ve yönetilmesini kolaylaştırır. Konteynerler, verileri izole bir şekilde tutar ve bu da veri bütünlüğünü sağlar.

\item Hızlı geri alma ve güncelleme: Docker, imaj ve konteyner tabanlı bir yaklaşım kullanarak hızlı geri alma ve güncelleme süreçleri sağlar. Eski bir imaja geri dönerek veya güncellenmiş bir imajla yeni bir konteyner oluşturarak sistemde yapılan değişiklikleri geri almak veya güncellemek kolaydır.

\item Kaynak verimliliği: Docker, paylaşılan çekirdek özelliklerini kullanarak kaynakların daha verimli bir şekilde kullanılmasını sağlar. Birçok konteyner, tek bir işletim sistemi çekirdeği üzerinde çalışır ve bu da sistem kaynaklarının daha etkin bir şekilde kullanılmasını sağlar.

\item Topluluk desteği: Docker, geniş bir kullanıcı ve geliştirici topluluğuna sahiptir. Bu topluluk, Docker hakkında bilgi paylaşımı, sorun giderme ve yenilikçi çözümler sunma konusunda yardımcı olur. Docker Hub gibi kaynaklar, hazır kullanıma uygun imajlar ve araçlar sağlar.
\end{itemize}

Docker'ın bu avantajları, uygulama geliştirme ve dağıtım süreçlerinde hızlılık, esneklik, güvenlik ve ölçeklenebilirlik sağlar. Ayrıca, Docker'ın genişletilebilirlik ve entegrasyon yetenekleri, farklı platformlar ve araçlarla uyumlu çalışmayı kolaylaştırır.

Neden Docker ?\\
Docker, bir yazılımın uygulama ve bağımlılıklarını bir "konteyner" olarak paketlemek ve çalıştırmak için kullanılan bir platformdur. Docker'ın sanal makinelerden farklı olarak konteynerizasyon teknolojisi üzerine kurulu olması, birkaç önemli avantaj sunmaktadır:\\
\begin{itemize}
\item Daha hafif ve hızlı: Sanal makineler (VM'ler), her biri kendi işletim sistemini çalıştıran ayrı bir sanal makine olduğundan, daha fazla kaynak tüketir ve yavaş çalışabilir. Docker konteynerleri ise ana işletim sistemini paylaşır ve sadece uygulama ve bağımlılıkları içerir. Bu nedenle, Docker konteynerleri daha hafif ve daha hızlıdır.\\

\item Daha verimli kaynak kullanımı: Docker, aynı fiziksel makine üzerinde birden çok konteyner çalıştırabilme yeteneği sayesinde kaynakları daha verimli kullanmanızı sağlar. Sanal makinelerde her bir VM, kendi işletim sistemini çalıştırdığından kaynak israfı yaşanabilirken, Docker konteynerleri aynı işletim sistemini paylaşarak daha etkili bir kaynak yönetimi sağlar.\\

\item Taşınabilirlik ve uyumluluk: Docker konteynerleri, uygulama ve bağımlılıklarını bir araya getirdiği için taşınabilirlik ve uyumluluk sağlar. Konteynerler, herhangi bir ortamda, herhangi bir makinede aynı şekilde çalışabilir. Bir kez oluşturulduklarında, Docker konteynerleri kolayca dağıtılabilir ve başka bir makineye taşınabilir.\\

\item Hızlı dağıtım ve ölçeklendirme: Docker konteynerleri, uygulamaların hızlı dağıtımını ve ölçeklendirilmesini sağlar. Konteynerlerin hızlı bir şekilde başlatılması ve durdurulması mümkündür. Ayrıca, bir uygulamayı birden çok konteyner olarak çalıştırarak yüksek kullanılabilirlik ve ölçeklenebilirlik elde etmek kolaydır.\\

\item İzolasyon ve güvenlik: Docker konteynerleri, birbirinden izole edilmiş çalışma ortamları sağlar. Her konteyner, kendi dosya sistemini, ağ bağlantılarını ve süreçlerini izole eder. Bu, bir konteynerin diğerlerinden etkilenmeyeceği anlamına gelir, böylece daha güvenli bir ortam sağlar.\\
\end{itemize}

\textbf{Docker vs Sanal Makianlar} \\
Sanal makine (VM) teknolojisi, ayrı işletim sistemleri ve bağımsız çalışma ortamları sağlar. Her bir sanal makine, kendi işletim sistemi, yazılım yığını ve kaynaklara sahip bir sanal bilgisayardır. VM'ler, fiziksel makine üzerinde çalışırken, her biri kendi kaynaklara sahip olduğundan izolasyon sağlarlar. Bu, farklı işletim sistemlerini aynı fiziksel makinede çalıştırmanın bir yolunu sunar. Ancak, VM'lerin daha fazla bellek, işlemci gücü ve depolama alanı gibi kaynaklara ihtiyaç duyması ve daha yavaş çalışması dezavantajlarından bazılarıdır.\\

Docker konteynerleri, hafif ve taşınabilir uygulama çalıştırma birimleridir. Docker, aynı çekirdek üzerinde çalışan ve paylaşılan kaynakları kullanan konteynerleştirilmiş uygulamaları yönetmek için bir platform sağlar. Her bir Docker konteyneri, izole bir çalışma ortamı sunar, ancak farklı bir işletim sistemi çalıştırmak yerine, ana makinedeki işletim sistemi ve çekirdek üzerinde çalışır. Bu, daha düşük bellek kullanımı ve daha hızlı başlatma süreleri gibi avantajlar sağlar. Ayrıca, Docker konteynerleri uygulama dağıtımını kolaylaştırır ve yazılımın farklı ortamlarda tutarlı bir şekilde çalışmasını sağlar.\\
Biz sanal makina olarak VirtualBox'ı baz alacağız.\\
\\\textbf{Docker vs VirtualBox}
\begin{table}[!h]
    \centering
    \begin{tabular}{|p{0.45\textwidth}|p{0.45\textwidth}|}
      \hline
      \textbf{Docker} & \textbf{VirtualBox} \\
      \hline
      Konteyner tabanlı sanallaştırma & Tam sanallaştırma \\
      \hline
      Aynı çekirdek üzerinde çalışır & Farklı işletim sistemlerini çalıştırır \\
      \hline
      Hafif ve taşınabilir konteynerler & Tam işletim sistemine sahip sanal makineler \\
      \hline
      İzole çalışma ortamı sağlar & İzole çalışma ortamı sağlar \\
      \hline
      Daha düşük bellek kullanımı ve hızlı başlatma & Daha fazla kaynak gerektirir \\
      \hline
      Uygulama dağıtımını kolaylaştırır & Farklı işletim sistemlerinin testi için kullanılır \\
      \hline
      Docker Hub'da konteyner görüntüleri paylaşabilir & Sanal makine görüntüleri kullanılabilir \\
      \hline
      Microservisler ve ölçeklenebilir uygulama geliştirmeleri & Özel izolasyon gerektiren durumlar \\
      \hline
    \end{tabular}
    \caption{Docker vs. VirtualBox}
  \end{table}
  
\subsection{Veri Tabanı İçin }
\subsubsection{MYSQL}
MySQL, açık kaynaklı bir ilişkisel veritabanı yönetim sistemidir. İlişkisel veritabanı yönetim sistemleri, verileri tablolarda saklayarak veriler arasındaki ilişkileri yönetmeyi sağlar. MySQL, istemci-sunucu modeline dayalı olarak çalışır ve bir sunucu üzerinde çalışan çok kullanıcılı bir veritabanı sistemidir.

MySQL, hızlı, güvenilir ve geniş bir kullanıcı tabanına sahip olmasıyla bilinir. Veritabanı yönetimi, veri manipülasyonu, sorgulama, güvenlik, yedekleme ve geri yükleme gibi birçok temel veritabanı işlemini destekler. Ayrıca, MySQL'in geniş bir uyumluluk listesi vardır ve farklı programlama dilleri ve platformlarla kolayca entegre olabilir.

MySQL, birçok web uygulamasında ve büyük ölçekli sistemlerde kullanılan yaygın bir veritabanı çözümüdür. Veri bütünlüğünü koruma, veritabanı güvenliği, yüksek performans, ölçeklenebilirlik ve kullanıcı dostu bir arayüz gibi önemli özelliklere sahiptir.

MySQL, veri tabanı yönetimi için güçlü ve esnek bir seçenek olup, çeşitli projelerde verilerin etkin bir şekilde saklanmasını ve yönetilmesini sağlar.

% resimler eklenecek 

